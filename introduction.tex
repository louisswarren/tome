Agda \citep{agdawiki} is a dependently-typed functional programming language
\citep{norellthesis}, based on an extension of intuitionistic Martin-L{\"o}f
type theory. We implement first order natural deduction in Agda. We use Agda's
type checker to verify the correctness natural deduction proofs, and also prove
properties of natural deduction, using Agda's proof assistant functionality.
This implementation corresponds to a formalisation of natural deduction in
constructive type theory, and the proofs are verified by Agda to be correct
(under the assumption that Agda itself is correct).

The Agda code below has been written in \emph{literate Agda}, which allows Agda
to be mixed with \LaTeX. The files which have been used to typeset this document
can also be evaluated and type checked. Some trivial proofs are omitted from the
typeset document; these are only hidden for brevity, and are still present in
the code and used by Agda. The results which rely on them are therefore still
verified. This should not be mistaken for use of postulates, wherein Agda itself
is told to assume that a proof exists. Postulates are used only in the module
for outputting natural deduction proofs as \LaTeX for use with the
\emph{bussproofs} package. All other code type checks with Agda in safe mode,
meaning that it provably halts.

Each of the following sections corresponds to its own literate Agda file.
Sections named with a file name ending in `.lagda' are modules. Each section
imports the modules preceding it, unless stated otherwise. These module
declarations and imports have been hidden for brevity.

Inspiration for the definition of vector types and decidable types comes from
the Agda standard library \citep{agdastdlib}. However, the standard library will
not be directly imported, to maintain clarity of definitions, and because it is
not needed. We will use built-in types for natural numbers, lists, and the
dependent sum, explaining their definitions when they appear.

The full code is available online at {\small \url{https://lsw.nz/tome}}.
